\subsection{1. Executive Summary}\label{executive-summary}}

Over the past decade, decentralized finance (DeFi) has emerged as one of
the most transformative movements in financial history. Yet, despite its
global ambitions, DeFi remains overwhelmingly denominated in U.S.
dollars, structurally excluding over 300 million European savers and
institutions who operate natively in euros. This asymmetry not only
exposes European participants to unnecessary currency risk but also
fails to capitalize on a vast pool of capital currently parked in
underperforming legacy products such as Livret A, life insurance
contracts, and regulated savings plans.

Quantillon Protocol seeks to rectify this imbalance through a
revolutionary three-token ecosystem: QEURO (euro-pegged stablecoin),
stQEURO (yield-bearing auto-compounding wrapper), and QTI (governance
token with vote-escrow mechanics). This synergistic architecture creates
interconnected value flows where each token reinforces the utility and
adoption of the others. QEURO operates through an innovative dual-pool
architecture---composed of "Users" who mint and stake QEURO, and
"Hedgers" who assume delta-neutral FX positions on the EUR/USD
pair---achieving fully decentralized, scalable, and compliant stablecoin
issuance. stQEURO automatically compounds yields from QEURO collateral
deployment without token inflation, offering Europeans their first
native yield-bearing euro instrument with instant liquidity and full
DeFi composability. QTI enables sophisticated governance through
vote-escrow mechanics, progressive decentralization, and sustainable
incentive alignment. At its core lies a set of innovative mechanisms:
permissionless hedging, yield redistribution through a variable "Yield
Shift," and modular collateral management through vault variants (e.g.,
aQEURO, mQEURO, bQEURO, eQEURO).

The fundamental economic insight underpinning Quantillon is an
application of the Cantillon Effect: in a financial ecosystem where
monetary expansion benefits capital allocators first, the lack of
euro-native DeFi instruments deprives European savers of fair access to
yield. The current euro financial infrastructure systematically
misallocates savings via opaque, fee-laden intermediaries and suboptimal
risk-reward profiles. Quantillon reimagines this ecosystem by offering a
highly liquid, capital-efficient, and decentralized alternative that can
serve both as a euro-denominated savings product and a trust-minimized
stablecoin.

From a macroeconomic standpoint, the project responds to a structural
divergence between the monetary policies of the Federal Reserve and the
European Central Bank. While the Fed\textquotesingle s aggressive
tightening cycle post-COVID has created a strong dollar carry trade, the
ECB has been slower to react, leaving the euro structurally weaker and
more volatile. This has made USD-denominated DeFi attractive globally,
but introduces a hedging need for eurozone participants---a need
Quantillon meets through embedded FX mechanisms.

Regulatorily, Quantillon stands apart through its alignment with MiCA
(Markets in Crypto Assets) regulation. By operating as a decentralized
protocol under Recital 22, it remains outside the scope of direct
regulatory enforcement while retaining pathways for institutional
compliance via the Quantillon Foundation. Early engagement with the
French ACPR (Autorité de Contrôle Prudentiel et de Résolution)
reinforces this legitimacy.

Economically, the protocol generates revenue through mint/redeem fees
(0.1\%) and by capturing a share of the yield generated from collateral
deployed on trusted DeFi platforms such as Aave. A conservative
projection at 20M TVL (Total Value Locked) with an average Aave APY of
7\% and swap volume 20x TVL yields over 500K EUR/year in protocol
revenue. This supports a scalable model with low fixed infrastructure
costs and exponential revenue potential.

Compared to existing euro-stablecoins such as EUROC (Circle), EURS
(Stasis), or EURT (Tether), Quantillon represents the only comprehensive
three-token euro-DeFi ecosystem, offering:

\begin{itemize}
\item
  Superior capital efficiency via delta-neutral hedging
\item
  Composability across multiple DeFi protocols
\item
  Liquidity by design through Forex and USDC markets
\item
  Yield-bearing QEURO instruments for both retail and institutional
  users
\end{itemize}

Finally, the protocol is backed by a seasoned team with decades of
experience in software architecture, financial systems, and DeFi
engineering. It is structured through three distinct but complementary
entities: Quantillon Protocol (on-chain governance), Quantillon Labs
(development and liquidity bootstrapping), and Quantillon Foundation
(regulatory interface).

In a European savings landscape marked by inertia, fragmentation, and
regulatory complexity, Quantillon Protocol presents a credible,
innovative, and technically rigorous solution. It is not merely a
stablecoin---but a new paradigm for euro-based digital savings and
decentralized capital formation.

\hypertarget{macro-regulatory-context}{%
\subsection{2. Macro \& Regulatory
Context}\label{macro-regulatory-context}}

\hypertarget{the-eurozones-monetary-landscape-a-fragile-equilibrium}{%
\subsubsection{2.1 The Eurozone's Monetary Landscape: A Fragile
Equilibrium}\label{the-eurozones-monetary-landscape-a-fragile-equilibrium}}

The European monetary union represents a unique economic construct: a
single currency deployed across divergent fiscal sovereignties. This
foundational asymmetry introduces persistent challenges in monetary
policy transmission, fiscal coordination, and currency stability. Since
its inception, the euro has oscillated between being a symbol of unity
and a structural constraint for its member states. Crucially, its
monetary architecture lacks a unified treasury, rendering the European
Central Bank (ECB) simultaneously omnipotent in setting interest rates
and impotent in addressing fiscal imbalances.

Following the COVID-19 pandemic, these weaknesses became more visible.
The ECB's asset purchase programs (APP, PEPP) ballooned its balance
sheet while interest rates remained negative or near zero for an
extended period. In parallel, the U.S. Federal Reserve embarked on a
rapid tightening cycle beginning in 2022, leading to a widening rate
differential between USD and EUR assets. The resulting capital outflows
from euro-denominated instruments depressed the euro's value,
culminating in a near-parity scenario with the U.S. dollar in mid-2022.

From a DeFi standpoint, this divergence has structural consequences.
With higher U.S. rates and deeper liquidity, DeFi protocols
overwhelmingly adopted USD as the reference currency. The euro became
underrepresented in decentralized markets, despite being the currency of
one of the world's largest economic blocs. This creates a double penalty
for European DeFi users: they face foreign exchange risk and lack native
euro-yielding instruments.

\hypertarget{the-case-for-euro-native-instruments-yield-safety-and-monetary-sovereignty}{%
\subsubsection{2.2 The Case for Euro-Native Instruments: Yield, Safety,
and Monetary
Sovereignty}\label{the-case-for-euro-native-instruments-yield-safety-and-monetary-sovereignty}}

European savers face a paradox. While they exhibit high precautionary
saving rates---estimated at over 12\% of household income across the
eurozone---most of this capital is deployed in vehicles offering
negligible returns: Livret A (France, 1.7\%), regulated life insurance
contracts, and pension products. These products are structurally
constrained by regulation, subject to opaque fee structures, and offer
limited capital appreciation.

At the macro level, the lack of euro-native yield instruments
exacerbates the Cantillon Effect. Monetary stimulus disproportionately
benefits financial intermediaries and capital holders close to the
central bank's monetary base. Retail savers, by contrast, receive only
the diluted remnants of yield transmission. This creates a structural
transfer from savers to asset managers---a pattern Quantillon explicitly
seeks to reverse.

By introducing QEURO as a euro-denominated, over-collateralized, and
yield-bearing stablecoin, the protocol offers a bridge between monetary
sovereignty and decentralized market participation. Users retain euro
exposure while accessing yields comparable to U.S. dollar strategies,
effectively equalizing global access to DeFi opportunities.

\hypertarget{regulatory-context-mica-and-the-exemption-framework}{%
\subsubsection{2.3 Regulatory Context: MiCA and the Exemption
Framework}\label{regulatory-context-mica-and-the-exemption-framework}}

The Markets in Crypto Assets (MiCA) regulation, set to be enforced
across the EU in 2025, aims to provide legal clarity and consumer
protection within crypto markets. However, it introduces stringent
requirements for stablecoins deemed "significant" or "asset-referenced."
These include capital reserves, whitepaper publication, supervisory
approval, and operational audits.

Recital 22 of MiCA introduces a critical exemption: protocols that are
``fully decentralized and not controlled by any legal entity'' fall
outside the regulation's purview. Quantillon leverages this exemption by
maintaining a trustless architecture: governance via the \$QTI token,
open-source smart contracts, and a non-custodial infrastructure.
Regulatory discussions with the French ACPR confirm the protocol's
compliance with this decentralized exemption.

Nevertheless, to enable institutional interaction, Quantillon
establishes the Quantillon Foundation. This entity holds governance
multisigs, interfaces with auditors, and can provide voluntary
disclosures or risk assessments. It bridges the gap between
decentralized governance and institutional requirements without
compromising protocol integrity.

This hybrid approach ensures both compliance and resilience. Quantillon
is designed to operate outside MiCA's scope while remaining
interoperable with it. It anticipates future legal harmonization and
sets a precedent for legally conscious DeFi architecture within the
European Union.

\hypertarget{market-landscape-competitive-analysis}{%
\subsection{3. Market Landscape \& Competitive
Analysis}\label{market-landscape-competitive-analysis}}

\hypertarget{the-euro-stablecoin-market-fragmentation-and-failure-to-launch}{%
\subsubsection{3.1 The Euro Stablecoin Market: Fragmentation and Failure
to
Launch}\label{the-euro-stablecoin-market-fragmentation-and-failure-to-launch}}

Despite numerous attempts to develop euro-denominated stablecoins, none
have achieved substantial market penetration or liquidity. EUROC
(Circle), EURS (Stasis), EURT (Tether), and Angle's EURA represent the
most recognizable names in this segment, yet they collectively represent
less than 1\% of the total stablecoin market capitalization as of Q2
2025. Each of these projects suffers from critical structural
limitations:

\begin{itemize}
\item
  \textbf{EUROC (Circle):} Despite the credibility of its issuer, EUROC
  has limited utility and suffers from low liquidity on decentralized
  exchanges. It also lacks yield generation mechanisms and is primarily
  used as a compliance showcase.
\item
  \textbf{EURS (Stasis):} EURS is a custodial stablecoin with limited
  adoption and high on/off-ramp costs. It is centralized and opaque,
  offering no native integration with DeFi protocols.
\item
  \textbf{EURT (Tether):} Lacking regulatory clarity and plagued by
  transparency concerns, EURT is largely avoided by institutional
  players.
\item
  \textbf{EURA (Angle Protocol):} While technically innovative with a
  dynamic reserve model, EURA has struggled with capital efficiency and
  scale, facing depegging events and low yield generation as it is based
  on euro bonds.
\end{itemize}

The consistent problem across all these assets is their failure to
create meaningful liquidity, yield opportunities, and DeFi integration
comparable to USD stablecoins. These failings result not from lack of
intent but from poor incentive alignment, limited network effects, and
insufficient hedging infrastructure for EUR/USD volatility.

\hypertarget{usd-dominance-in-defi-a-structural-bias}{%
\subsubsection{3.2 USD Dominance in DeFi: A Structural
Bias}\label{usd-dominance-in-defi-a-structural-bias}}

As of mid-2025, over 99\% of DeFi stablecoin liquidity is denominated in
USD. The dominance of USDC and USDT on platforms like Aave, Compound,
Curve, and MakerDAO creates a feedback loop: most lending, borrowing,
and LP positions rely on dollar-based units of account. This has network
and liquidity advantages but perpetuates the exclusion of non-USD
participants.

From a European perspective, engaging in DeFi through USD-based assets
introduces three layers of friction:

\begin{enumerate}
\def\labelenumi{\arabic{enumi}.}
\item
  \begin{quote}
  \textbf{FX risk} -- EUR/USD fluctuations can nullify yield gains or
  exacerbate losses.
  \end{quote}
\item
  \begin{quote}
  \textbf{Operational slippage} -- On-ramping EUR into USD-based DeFi
  typically requires high-friction conversions via centralized
  exchanges.
  \end{quote}
\item
  \begin{quote}
  \textbf{Regulatory and tax complexity} -- Cross-currency gains may
  introduce additional accounting burdens and reduce fiscal clarity.
  \end{quote}
\end{enumerate}

Consequently, euro-based users and institutions either remain absent
from DeFi altogether or engage through inefficient intermediaries. This
creates an untapped market segment, particularly among family offices,
corporate treasuries, and fintech platforms seeking native euro
liquidity.

\hypertarget{quantillons-competitive-edge}{%
\subsubsection{3.3 Quantillon's Competitive
Edge}\label{quantillons-competitive-edge}}

Quantillon Protocol is positioned to resolve the failures of previous
euro stablecoins by aligning incentives and market architecture through
three innovations:

\begin{itemize}
\item
  \textbf{Liquidity by Design:} QEURO inherits USDC liquidity on the
  user side and Forex liquidity via the hedger side. This dual-channel
  design mitigates the need for external liquidity mining or bribing
  mechanisms.
\item
  \textbf{Delta-Neutral Hedging:} Unlike existing euro stablecoins,
  Quantillon introduces a permissionless hedging layer that allows
  collateral providers to neutralize EUR/USD exposure through
  protocol-native instruments. This supports peg stability and
  institutional hedging needs.
\item
  \textbf{Yield Shift Mechanism:} QEURO is not only a stablecoin but a
  savings instrument. By redistributing most of the yield from
  collateral deployment (e.g., Aave) to users and hedgers via a dynamic
  `Yield Shift', the protocol incentivizes long-term participation and
  peg maintenance.
\end{itemize}

These components create a uniquely sustainable design that addresses
both supply (hedgers) and demand (euro users) sides of the market. By
leveraging DeFi primitives and real-world financial theory---including
FX swap economics and interest rate parity---Quantillon delivers a euro
stablecoin that is liquid, scalable, and yield-generating.

In short, Quantillon does not merely replicate a euro version of USDC or
DAI---it reengineers stablecoin economics from the ground up to serve
European needs in a structurally USD-centric DeFi landscape.

\hypertarget{quantillon-protocol-design-architecture}{%
\subsection{4. Quantillon Protocol: Design \&
Architecture}\label{quantillon-protocol-design-architecture}}

\hypertarget{qeuro-stablecoin-design-a-euro-pegged-overcollateralized-instrument}{%
\subsubsection{4.1 QEURO Stablecoin Design: A Euro-Pegged,
Overcollateralized
Instrument}\label{qeuro-stablecoin-design-a-euro-pegged-overcollateralized-instrument}}

At the core of the Quantillon Protocol lies the QEURO---a euro-pegged
stablecoin minted against overcollateralized reserves in USD
stablecoins, initially USDC. The choice of overcollateralization offers
a high-integrity approach to maintaining the peg and minimizing
counterparty risk.

QEURO is minted permissionlessly at the oracle price, allowing any user
to deposit accepted ERC20 assets which are then routed through the
protocol's smart contracts to be swapped into USDC and locked as
collateral. The protocol ensures a minimum 101\% collateralization
ratio, with liquidation mechanisms triggered automatically via Chainlink
or equivalent high-reliability oracles if the threshold is breached.
Redeeming QEURO follows the same logic in reverse, enabling frictionless
exit at minimal cost.

The result is a stablecoin that provides euro exposure with USD-based
liquidity depth, solving a crucial mismatch in DeFi's monetary
architecture.

\hypertarget{the-dual-pool-model-users-and-hedgers}{%
\subsubsection{4.2 The Dual-Pool Model: Users and
Hedgers}\label{the-dual-pool-model-users-and-hedgers}}

Quantillon introduces a dual-pool mechanism involving two symbiotic
actor groups: \textbf{Users} and \textbf{Hedgers}.

\begin{itemize}
\item
  \textbf{Users} are DeFi participants who mint QEURO to access
  euro-denominated liquidity or stake QEURO in order to earn yield. They
  deposit volatile or stable crypto assets, which are swapped into USDC
  and locked as protocol collateral. These users maintain exposure to
  the euro while benefiting from DeFi-native financial products.
\item
  \textbf{Hedgers} are actors who take the opposite position---providing
  USDC upfront to hedge against euro exposure. They effectively short
  the EUR/USD pair by engaging in a leveraged, unidirectional perpetual
  future. In return, they earn compensation consisting of (i) the
  EUR/USD interest rate differential (typically \textasciitilde1\%
  annually) and (ii) a variable "Yield Shift" bonus extracted from the
  collateral's DeFi yield.
\end{itemize}

The protocol enforces a liquidation mechanism for hedgers as well: if
their margin ratio drops below 1\%, they are liquidated and penalized.
This mechanism ensures disciplined FX exposure management and peg
resilience.

\hypertarget{liquidity-architecture-inheriting-depth-from-usdc-and-forex}{%
\subsubsection{4.3 Liquidity Architecture: Inheriting Depth from USDC
and
Forex}\label{liquidity-architecture-inheriting-depth-from-usdc-and-forex}}

Unlike euro-stablecoin predecessors that relied heavily on centralized
market makers or illiquid LP incentives, Quantillon's model ensures
"liquidity by design."

\begin{itemize}
\item
  \textbf{On the user side}, all incoming deposits are swapped into
  USDC---the most liquid dollar-based stablecoin---thus ensuring instant
  deployability and deep routing.
\item
  \textbf{On the hedger side}, liquidity stems from traditional Forex
  markets, where EUR/USD is one of the most traded currency pairs
  globally. This design enables the protocol to scale capital efficiency
  without relying on shallow DEX pools or bribe-driven gauges.
\end{itemize}

As a result, QEURO enjoys both upstream (USDC) and downstream (Forex)
liquidity, reinforcing slippage-free mint/redeem operations and improved
user experience.

\hypertarget{native-composability-and-embedded-dex}{%
\subsubsection{4.4 Native Composability and Embedded
DEX}\label{native-composability-and-embedded-dex}}

Quantillon is designed as a holistically integrated protocol stack.
Rather than relying on external DEXs like Curve or Balancer, the
protocol features its own embedded exchange mechanism. This ensures
control over routing fees, eliminates bribing systems, and increases
revenue retention.

Smart contract logic for mint/redeem, hedging, staking, liquidation, and
governance is unified under a modular Solidity-based architecture
audited and battle-tested through external bug bounty programs.
Governance functions, including Yield Shift calibration and vault
whitelisting, are performed by holders of the \$QTI token.

\hypertarget{vault-variants-modular-architecture-for-risk-segmentation}{%
\subsubsection{4.5 Vault Variants: Modular Architecture for Risk
Segmentation}\label{vault-variants-modular-architecture-for-risk-segmentation}}

Quantillon\textquotesingle s modular design enables multiple vault
variants, each corresponding to specific collateral backends and risk
profiles. This architecture serves diverse user needs while maintaining
unified euro peg mechanics across all variants.

\textbf{Vault Portfolio Overview}

\begin{longtable}[]{@{}
  >{\raggedright\arraybackslash}p{(\columnwidth - 8\tabcolsep) * \real{0.1374}}
  >{\raggedright\arraybackslash}p{(\columnwidth - 8\tabcolsep) * \real{0.3407}}
  >{\raggedright\arraybackslash}p{(\columnwidth - 8\tabcolsep) * \real{0.1516}}
  >{\raggedright\arraybackslash}p{(\columnwidth - 8\tabcolsep) * \real{0.1422}}
  >{\raggedright\arraybackslash}p{(\columnwidth - 8\tabcolsep) * \real{0.2281}}@{}}
\toprule\noalign{}
\begin{minipage}[b]{\linewidth}\raggedright
Vault Type
\end{minipage} & \begin{minipage}[b]{\linewidth}\raggedright
Collateral Backend
\end{minipage} & \begin{minipage}[b]{\linewidth}\raggedright
Risk Profile
\end{minipage} & \begin{minipage}[b]{\linewidth}\raggedright
Target APY
\end{minipage} & \begin{minipage}[b]{\linewidth}\raggedright
Target Users
\end{minipage} \\
\midrule\noalign{}
\endhead
\bottomrule\noalign{}
\endlastfoot
\textbf{aQEURO} & Aave USDC lending & 🟢 Low & 4-8\% & Retail
investors \\
\textbf{mQEURO} & MakerDAO PSM/DSR & 🟢 Very Low & 3-6\% & Conservative
users \\
\textbf{bQEURO} & Tokenized T-Bills \& RWAs & 🟢 Low & 5-7\% &
Institutions \\
\textbf{eQEURO} & Ethena \& advanced strategies & 🟡 Medium & 6-12\% &
Sophisticated DeFi \\
\end{longtable}

\textbf{Detailed Vault Specifications}

aQEURO (Aave Integration - Default Configuration)

\begin{itemize}
\item
  Collateral Deployment: Aave USDC lending
\item
  Yield Mechanism: Direct participation in Aave lending markets with
  dynamic APY
\item
  Risk Factors: Aave protocol risk, USDC depeg risk, smart contract risk
\item
  Target Users: Retail DeFi users seeking stable euro yield with proven
  infrastructure
\end{itemize}

mQEURO (MakerDAO Integration)

\begin{itemize}
\item
  Collateral Deployment: MakerDAO Peg Stability Module (PSM) and Dai
  Savings Rate (DSR)
\item
  Yield Mechanism: Conservative yield through DAI ecosystem stability
  mechanisms
\item
  Risk Factors: MakerDAO governance risk, DAI stability risk
\item
  Target Users: Ultra-conservative users prioritizing capital
  preservation
\end{itemize}

bQEURO (Real World Assets Focus)

\begin{itemize}
\item
  Collateral Deployment: Tokenized U.S. Treasury Bills, European Bunds,
  Corporate Bonds
\item
  Yield Mechanism: Traditional fixed-income yields through compliant
  tokenization platforms
\item
  Risk Factors: Custody risk, regulatory risk, lower liquidity
\item
  Target Users: Institutional investors requiring regulatory clarity and
  traditional yield sources
\item
  Minimum Threshold: €100,000 institutional access requirement
\end{itemize}

eQEURO (Advanced DeFi Strategies)

\begin{itemize}
\item
  Collateral Deployment: Ethena USDe, Lido stETH, Rocketpool rETH,
  recursive strategies
\item
  Yield Mechanism: Leveraged yield farming up to 3x through
  sophisticated DeFi composability
\item
  Risk Factors: Higher smart contract risk, leverage risk, strategy
  complexity
\item
  Target Users: Sophisticated DeFi participants seeking maximum yield
  optimization
\item
  Advanced Features: Automated rebalancing, yield optimization bots,
  governance-adjustable parameters
\end{itemize}

\textbf{Cross-Vault Mechanics}

Unified Peg Maintenance: All vault variants maintain the same EUR target
price through shared oracle infrastructure and arbitrage mechanisms.

Risk Isolation: Individual vault failures are contained and do not
affect other variants through modular smart contract architecture.

Arbitrage Opportunities: Price differences between variants create
natural trading incentives that support overall ecosystem liquidity.

Governance Coordination: QTI holders vote on vault-specific parameters
while maintaining ecosystem-wide coherence.

\hypertarget{stqeuro-yield-bearing-euro-infrastructure}{%
\subsubsection{4.6 stQEURO~: Yield-Bearing Euro
Infrastructure}\label{stqeuro-yield-bearing-euro-infrastructure}}

Beyond serving as a euro-pegged stablecoin, Quantillon introduces
stQEURO---a yield-bearing wrapper that automatically compounds returns
from QEURO collateral deployment while maintaining full liquidity and
composability. This innovation bridges the gap between passive euro
exposure and active DeFi yield generation.

\textbf{Auto-Compounding Mechanism}

Unlike traditional staking systems requiring manual reward claims,
stQEURO automatically increases in intrinsic value over time. The
quantity of stQEURO tokens in user wallets remains constant, but each
token becomes worth more QEURO as yields accumulate:

\textbf{Value Appreciation Formula:}

stQEURO Exchange Rate = 1 + (Cumulative Yield ÷ Total Days × Days Held)

Example Timeline:

Day 0: Stake 1,000 QEURO → Receive 1,000 stQEURO (Rate: 1.000)

Day 365: Still hold 1,000 stQEURO (Rate: 1.053 with 5.3\% APY)

Unstaking: 1,000 stQEURO → 1,053 QEURO

\textbf{Yield Source \& Distribution}

stQEURO\textquotesingle s yield originates from the underlying QEURO
collateral deployed across battle-tested DeFi protocols:

Yield Distribution: Protocol Fees: 10\% (Treasury + Development)

Hedger Compensation: Variable 1-3\% (EUR/USD spread + Yield Shift)

stQEURO Holders: Remaining 87-89\% (Auto-compounded)

\textbf{Technical Advantages}

Instant Liquidity: stQEURO maintains full liquidity without lock
periods, enabling immediate unstaking at current exchange rates.

DeFi Composability: Users can deploy stQEURO in other protocols
(lending, LP positions, collateral) while continuing to earn compounding
yield.

Tax Efficiency: No rebase events or reward distributions that could
trigger taxable income in many jurisdictions.

MEV Protection: Built-in front-running resistance for staking operations
through oracle-based pricing and transaction batching.

\hypertarget{tokenomics-governance}{%
\subsection{5. Tokenomics \& Governance}\label{tokenomics-governance}}

\hypertarget{the-qti-token-governance-distribution-and-incentives}{%
\subsubsection{5.1 The \$QTI Token: Governance, Distribution, and
Incentives}\label{the-qti-token-governance-distribution-and-incentives}}

Quantillon operates through a sophisticated three-token system designed
to create sustainable value flows and optimal capital efficiency across
the euro-DeFi ecosystem.

\textbf{The \$QTI Token: Governance \& Value Accrual}

The \$QTI token serves as the governance backbone of the Quantillon
Protocol, featuring advanced vote-escrow (veQTI) mechanics and
progressive decentralization. With a fixed total supply of 100,000,000
QTI tokens, the distribution is strategically structured as follows:

\textbf{Strategic Token Allocation:}

\begin{longtable}[]{@{}
  >{\raggedright\arraybackslash}p{(\columnwidth - 8\tabcolsep) * \real{0.2637}}
  >{\raggedright\arraybackslash}p{(\columnwidth - 8\tabcolsep) * \real{0.1236}}
  >{\raggedright\arraybackslash}p{(\columnwidth - 8\tabcolsep) * \real{0.1621}}
  >{\raggedright\arraybackslash}p{(\columnwidth - 8\tabcolsep) * \real{0.1374}}
  >{\raggedright\arraybackslash}p{(\columnwidth - 8\tabcolsep) * \real{0.3133}}@{}}
\toprule\noalign{}
\begin{minipage}[b]{\linewidth}\raggedright
Category
\end{minipage} & \begin{minipage}[b]{\linewidth}\raggedright
Allocation
\end{minipage} & \begin{minipage}[b]{\linewidth}\raggedright
Amount
\end{minipage} & \begin{minipage}[b]{\linewidth}\raggedright
Lock Period
\end{minipage} & \begin{minipage}[b]{\linewidth}\raggedright
Vesting Schedule
\end{minipage} \\
\midrule\noalign{}
\endhead
\bottomrule\noalign{}
\endlastfoot
\textbf{Community \& Ecosystem} & 50\% & 50,000,000 QTI & Variable &
48-month algorithmic curve \\
\textbf{Team \& Founders} & 15\% & 15,000,000 QTI & 12 months & 36
months linear \\
\textbf{Investors (SAFT/BSA)} & 13\% & 13,000,000 QTI & 6-18 months &
24-36 months tiered \\
\textbf{DAO Treasury} & 10\% & 10,000,000 QTI & Immediate &
Governance-controlled \\
\textbf{Strategic Partners} & 5\% & 5,000,000 QTI & 6 months & 18 months
performance-based \\
\textbf{Advisors} & 2\% & 2,000,000 QTI & 6 months & 18 months
milestone-driven \\
\textbf{Liquidity Provision} & 5\% & 5,000,000 QTI & Immediate &
Market-responsive release \\
\end{longtable}

\hypertarget{section-1}{%
\paragraph{}\label{section-1}}

\textbf{Vote-Escrow (veQTI) System}

QTI holders can lock their tokens for periods ranging from 1 week to 4
years, receiving voting power multipliers up to 4x base weight. This
system ensures long-term alignment and prevents governance attacks while
enabling meaningful decentralized decision-making.

\textbf{Governance operates through three layers:}

\begin{itemize}
\item
  \textbf{Constitutional Changes:} 85\% threshold, 14-day timelock
  (protocol parameters, emergency procedures)
\item
  \textbf{Operational Decisions:} 60\% threshold, 3-day timelock (fee
  structures, incentive programs)
\item
  \textbf{Community Proposals:} Simple majority, 24-hour timelock
  (grants, marketing initiatives)
\end{itemize}

\hypertarget{section-2}{%
\paragraph{}\label{section-2}}

\textbf{stQEURO: Yield-Bearing Euro Infrastructure}

stQEURO represents the protocol\textquotesingle s yield-bearing token,
automatically compounding returns from QEURO collateral deployment.
Unlike traditional staking mechanisms, stQEURO maintains constant token
quantity in user wallets while increasing intrinsic value over time
through the formula:

\textbf{stQEURO Value = 1 stQEURO = (1 + Cumulative Yield Rate) QEURO}

Key benefits include:

\begin{itemize}
\item
  \textbf{Automatic Compounding:} No manual reinvestment required
\item
  \textbf{Instant Liquidity:} No lock periods or withdrawal delays
\item
  \textbf{DeFi Composability:} Full integration across protocols while
  earning yield
\item
  \textbf{Tax Efficiency:} No rebase events creating potential taxable
  income
\end{itemize}

\hypertarget{section-3}{%
\paragraph{}\label{section-3}}

\hypertarget{yield-mechanics-and-the-yield-shift}{%
\subsubsection{5.2 Yield Mechanics and the "Yield
Shift"}\label{yield-mechanics-and-the-yield-shift}}

Quantillon introduces an innovative mechanism called the \textbf{Yield
Shift}, which serves as the protocol's internal rebalancing engine. It
functions by redistributing yield between Users and Hedgers based on
market conditions, supply/demand imbalances, and peg deviation
pressures.

Collateral deployed in DeFi protocols (e.g., Aave) generates a baseline
APY. From this yield:

\begin{itemize}
\item
  A protocol fee of 10\% is applied.
\item
  Hedgers receive a fixed compensation based on the EUR/USD interest
  rate spread (typically \textasciitilde1\%).
\item
  The residual is then distributed variably:

  \begin{itemize}
  \item
    \textbf{Positive Yield Shift}: More yield incentivizes Hedgers when
    their supply is insufficient.
  \item
    \textbf{Negative Yield Shift}: More yield flows to Users when
    hedgers participation is high.
  \end{itemize}
\end{itemize}

This creates a dynamic equilibrium. The Yield Shift is not
discretionary; it is governed by predefined formulas based on real-time
FX rate and User/Hedger supply and demand. Governance can only modify
its parameters within capped ranges, preserving systemic integrity.

\hypertarget{incentive-alignment-and-protocol-sustainability}{%
\subsubsection{5.3 Incentive Alignment and Protocol
Sustainability}\label{incentive-alignment-and-protocol-sustainability}}

\$QTI also serves as an incentive layer through \textbf{liquidity mining
programs}, staking multipliers, and governance rewards. These incentives
are time-bound and designed to bootstrap early adoption without creating
long-term inflationary pressures.

In the longer term, the protocol aims to activate the \textbf{Fee
Switch}, diverting a portion of transaction and yield fees to a treasury
governed by \$QTI holders. This treasury may be used to:

\begin{itemize}
\item
  Fund audits and research
\item
  Provide insurance buffers
\item
  Invest in ecosystem integrations or cross-chain bridges
\end{itemize}

Sustainability is further ensured by the protocol's lean cost structure.
With an estimated burn rate of €400,000 per year and projected revenues
of €500,000 at 20M TVL, Quantillon achieves operating surplus early on.
This surplus can be reinvested in growth or redistributed through token
buybacks or veQTI-style locking mechanisms.

Governance mechanisms are engineered to be progressive. In early phases,
protocol changes may require multi-signature validation from the
Quantillon Foundation to ensure operational security. Over time, power
will transition toward full DAO control, contingent on metrics like TVL,
QTI token dispersion, and governance participation rates.

Quantillon's tokenomics combine strong economic incentives with a
governance architecture inspired by proven DeFi protocols such as Curve,
Aave, and MakerDAO, while adapting them to the specific needs of
eurozone compliance and FX stability.

\hypertarget{business-model-economic-sustainability}{%
\subsection{6. Business Model \& Economic
Sustainability}\label{business-model-economic-sustainability}}

Quantillon Protocol is structured around a dual-source revenue model
that prioritizes scalability, capital efficiency, and long-term
sustainability. In contrast to speculative or transaction-based DeFi
projects, Quantillon generates recurring income from two distinct but
complementary streams:

\begin{itemize}
\item
  \textbf{Protocol fees:} A fixed 0.1\% fee is levied on all mint and
  redeem operations involving QEURO.
\item
  \textbf{Collateral yield:} The protocol captures 10\% of the yield
  generated by deploying collateral (primarily USDC) in battle-tested
  DeFi platforms such as Aave.
\end{itemize}

This model is transparent, predictable, and resilient across market
cycles. It aligns closely with the utility-driven design of the
protocol, ensuring that revenues scale proportionally with user activity
and TVL (Total Value Locked).

\hypertarget{revenue-forecasts-and-operating-leverage}{%
\subsubsection{6.1 Revenue Forecasts and Operating
Leverage}\label{revenue-forecasts-and-operating-leverage}}

Quantillon\textquotesingle s three-token architecture generates multiple
sustainable revenue streams that scale proportionally with ecosystem
growth and user adoption.

\textbf{Primary Revenue Sources}

\textbf{1. QEURO Operations (Core Stablecoin Activity)}

\begin{itemize}
\item
  Mint/redeem fees: 0.1\% on all QEURO operations
\item
  Yield management: 10\% of returns from collateral deployment
  (primarily Aave USDC)
\item
  Liquidation penalties: 2-5\% of liquidated hedger and user positions
\end{itemize}

\textbf{2. stQEURO Yield Infrastructure}

\begin{itemize}
\item
  Staking operation fees: 0.05\% on stake/unstake transactions
\item
  Enhanced trading volume through improved user retention
\item
  Premium yield optimization services for institutional users
\end{itemize}

\textbf{3. Cross-Protocol Integration}

\begin{itemize}
\item
  Cross-chain bridge fees: 0.1-0.3\% of transfers (Ethereum, Base,
  Arbitrum, Optimism)
\item
  Premium institutional services: advanced analytics, custom
  integrations
\item
  Partnership revenue sharing: integrations with CeDeFi platforms
\end{itemize}

Conservative Growth Scenario (€50M TVL by Year 1)

\ul{Revenue Breakdown:}

Annual Revenue Calculation:

\begin{itemize}
\item
  Swap Volume: €500M (10x TVL turnover)
\item
  Swap Fees: €500M × 0.1\% = €500K
\item
  Yield Management: €50M × 7\% × 10\% = €350K
\item
  stQEURO Operations: €20M staked × 2\% annual turnover × 0.05\% = €20K
\item
  Cross-Chain Fees: €100M volume × 0.2\% average = €200K
\item
  Liquidation Penalties: €2M volume × 3\% average = €60K
\item
  Total Annual Revenue: €1,130K
\item
  Operating Costs: €600K (development, infrastructure, legal, audits)
\item
  Net Operating Profit: €530K (47\% margin)
\end{itemize}

Optimistic Growth Scenario (€500M TVL by Year 2)

\ul{Revenue Breakdown:}

Annual Revenue Calculation:

\begin{itemize}
\item
  Swap Volume: €5B (10x TVL turnover)
\item
  Swap Fees: €5B × 0.1\% = €5M
\item
  Yield Management: €500M × 7\% × 10\% = €3.5M
\item
  stQEURO Operations: €300M staked × 3\% annual turnover × 0.05\% =
  €450K
\item
  Cross-Chain Fees: €1.5B volume × 0.2\% average = €3M
\item
  Liquidation Penalties: €50M volume × 3\% average = €1.5M
\item
  Total Annual Revenue: €13.45M
\item
  Operating Costs: €2.5M (scaled operations)
\item
  Net Operating Profit: €10.95M (81\% margin)
\end{itemize}

\textbf{Key Performance Indicators}

\textbf{Growth Metrics:}

\begin{itemize}
\item
  TVL Growth: Target €100M by Month 12, €1B by Month 36
\item
  Staking Adoption: 50\%+ of QEURO supply in stQEURO by Year 2
\item
  Daily Volume: 2-5\% of TVL in trading activity
\item
  Cross-Chain Distribution: 30\% mainnet, 70\% L2 by Year 2
\end{itemize}

\textbf{Sustainability Metrics:}

\begin{itemize}
\item
  Operating Margin: Maintain \textgreater40\% across market conditions
\item
  Revenue Diversification: No single source \textgreater60\% of total
  revenue
\item
  User Retention: \textgreater80\% of stQEURO holders active after 6
  months
\item
  Protocol Utilization: Average \textgreater80\% of collateral deployed
  in yield strategies
\end{itemize}

\hypertarget{capital-efficiency-through-design}{%
\subsubsection{6.2 Capital Efficiency Through
Design}\label{capital-efficiency-through-design}}

Quantillon's model avoids several common inefficiencies in DeFi
protocols:

\begin{itemize}
\item
  No dependence on liquidity mining for peg stability
\item
  No bribing mechanisms for gauge voting or yield direction
\item
  Slippage-free mint/redeem operations eliminate arbitrage cost
\end{itemize}

The dual-pool architecture with dynamic Yield Shift ensures internal
market balance, reducing the need for external incentives or
unsustainable emissions. This design is particularly adapted to the
eurozone context, where investors tend to prioritize capital
preservation and yield visibility.

\hypertarget{institutional-fit-and-monetization-pathways}{%
\subsubsection{6.3 Institutional Fit and Monetization
Pathways}\label{institutional-fit-and-monetization-pathways}}

Quantillon's infrastructure allows for multiple monetization pathways:

\begin{itemize}
\item
  \textbf{Institutional access:} Hedge funds, wealth managers, and
  corporate treasuries can use QEURO for euro-native exposure in DeFi
  with reduced compliance overhead.
\item
  \textbf{CeDeFi integration:} Through partnerships with platforms like
  Quantfury or Cadmos, QEURO can serve as the euro liquidity backbone in
  compliant fintech stacks.
\item
  \textbf{On/off-ramp monetization:} Future integrations with fiat
  gateways and custody providers may allow for spread-based revenue
  models.
\end{itemize}

The protocol also supports cross-chain deployment, allowing it to tap
into ecosystems like Arbitrum, Base, or LayerZero for added liquidity
depth and composability. Treasury diversification and fixed-income DeFi
strategies can further enhance yield capture without increasing risk.

In sum, Quantillon is engineered as a low-cost, high-leverage protocol.
It combines the efficiency of smart contract automation with the
robustness of euro-denominated asset management, resulting in a rare
alignment of financial viability and technical rigor in the DeFi space.

\hypertarget{risks-mitigation-strategies}{%
\subsection{7. Risks \& Mitigation
Strategies}\label{risks-mitigation-strategies}}

\hypertarget{currency-risk-eurusd-volatility}{%
\subsubsection{7.1 Currency Risk: EUR/USD
Volatility}\label{currency-risk-eurusd-volatility}}

Quantillon\textquotesingle s core innovation---serving euro users via
USD-collateralized instruments---requires active hedging of EUR/USD
exposure. Volatility in this currency pair, especially under divergent
monetary policies between the ECB and the Fed, presents a persistent
risk to peg maintenance.

\textbf{Mitigation:} The protocol's architecture is explicitly
delta-neutral. Hedgers assume long EUR/USD positions in exchange for
predictable compensation, aligning their incentives with peg stability.
Liquidation thresholds and real-time FX oracles (e.g., Chainlink)
enforce discipline. Additionally, the Yield Shift mechanism acts as a
buffer: as FX volatility rises, incentives for hedgers increase
dynamically.

\hypertarget{liquidity-risk-capital-flight-and-redemption-pressure}{%
\subsubsection{7.2 Liquidity Risk: Capital Flight and Redemption
Pressure}\label{liquidity-risk-capital-flight-and-redemption-pressure}}

Periods of market stress may trigger mass redemptions, potentially
challenging the protocol's ability to unwind positions or liquidate
collateral efficiently.

\textbf{Mitigation:} Quantillon inherits the deep liquidity of USDC on
the user side and leverages the Forex market on the hedger side.
Redemption operations are slippage-free, and collateral is deployed on
liquid DeFi markets (e.g., Aave) with short withdrawal queues.
Furthermore, vault variants allow diversification of exposure (e.g.,
T-Bills, Maker vaults), improving redemption resiliency.

\hypertarget{smart-contract-and-oracle-risk}{%
\subsubsection{7.3 Smart Contract and Oracle
Risk}\label{smart-contract-and-oracle-risk}}

The protocol relies on smart contracts for collateral management,
minting, and liquidation. Any vulnerability---whether in protocol
contracts or oracle feeds---can undermine systemic integrity.

\textbf{Mitigation:} All core smart contracts undergo rigorous auditing
and continuous bug bounty programs. The protocol adopts a modular
architecture, limiting systemic blast radius in case of an exploit.
Oracles are sourced from multiple providers and include circuit breakers
for anomalous readings.

\hypertarget{governance-risk-and-protocol-capture}{%
\subsubsection{7.4 Governance Risk and Protocol
Capture}\label{governance-risk-and-protocol-capture}}

As with all DAO-based systems, Quantillon faces the risk of governance
capture or low voter participation, especially in early phases.

\textbf{Mitigation:} Governance powers are progressively decentralized.
Initially, the Quantillon Foundation holds veto rights over critical
upgrades to ensure stability. Over time, governance evolves toward \$QTI
token holders. A quorum-based system and time-locked proposals provide
transparency and delay in decision-making.

\hypertarget{regulatory-risk-legal-classification-and-mica}{%
\subsubsection{7.5 Regulatory Risk: Legal Classification and
MiCA}\label{regulatory-risk-legal-classification-and-mica}}

Although Quantillon benefits from the Recital 22 exemption under MiCA,
regulatory interpretation can evolve, especially as EU authorities
refine crypto oversight.

\textbf{Mitigation:} The protocol operates under a hybrid compliance
architecture. Its decentralized nature is verifiable and publicly
auditable. Meanwhile, the Quantillon Foundation interfaces with
regulators and external auditors, maintaining legal dialogue (notably
with the French ACPR). The Foundation can issue voluntary disclosures or
risk assessments without compromising decentralization.

\hypertarget{defi-contagion-risk}{%
\subsubsection{7.6 DeFi Contagion Risk}\label{defi-contagion-risk}}

Quantillon interacts with other DeFi platforms, notably Aave. In the
event of failure or depegging on these platforms, collateral could be
impaired.

\textbf{Mitigation:} Collateral is diversified across whitelisted
protocols. Vault variants can be adjusted via governance to reduce
exposure. The protocol monitors real-time risk parameters, including
liquidity ratios and asset volatility. In extreme scenarios, emergency
pause mechanisms are available.

Quantillon is built on the principle of resilient decentralization.
Rather than avoiding risk, it manages it explicitly through incentive
design, robust engineering, and layered governance. This makes it
uniquely equipped to operate in a volatile, fragmented, and evolving
DeFi environment.

\hypertarget{stqeuro-specific-risk-factors}{%
\subsubsection{7.7 stQEURO-Specific Risk
Factors}\label{stqeuro-specific-risk-factors}}

The introduction of yield-bearing euro infrastructure creates additional
risk vectors requiring specialized mitigation strategies.

\textbf{Auto-Compounding Mechanism Risk}

Risk: Smart contract vulnerabilities in yield calculation or
distribution logic could result in incorrect stQEURO valuations or user
fund loss.

Mitigation:

\begin{itemize}
\item
  Specialized audits focused exclusively on auto-compounding token
  mechanics
\item
  Formal mathematical verification of yield calculation algorithms
\item
  Real-time monitoring with automated circuit breakers for anomalous
  calculations
\item
  Emergency pause functionality with governance-controlled restart
  procedures
\item
  Segregated insurance fund (3\% of yield) specifically for compounding
  mechanism failures
\end{itemize}

\textbf{Yield Volatility and User Expectation Risk}

Risk: Rapid changes in underlying Aave yields could create user
dissatisfaction or mass unstaking events if stQEURO returns fall below
expectations.

Mitigation:

\begin{itemize}
\item
  7-day moving average smoothing for APY display to reduce short-term
  volatility perception
\item
  5\% protocol yield reserve buffer to maintain consistent distributions
  during temporary yield drops
\item
  Maximum 15\% APY ceiling to prevent unsustainable user expectations
  during yield spikes
\item
  Transparent communication about yield sources and market dependency
\end{itemize}

\textbf{Cross-Protocol Composability Risk}

Risk: stQEURO\textquotesingle s use as collateral in other DeFi
protocols could create cascading liquidation events if stQEURO value
appreciation calculations fail or if oracle feeds become unreliable.

Mitigation:

\begin{itemize}
\item
  Conservative oracle update mechanisms with multiple data sources and
  heartbeat monitoring
\item
  Partnership agreements with major DeFi protocols for standardized
  stQEURO valuation methods
\item
  Graduated rollout of stQEURO integrations with smaller protocols
  before major platform adoption
\item
  Emergency communication channels for rapid coordination during
  cross-protocol incidents
\end{itemize}

\hypertarget{roadmap-adoption-strategy}{%
\subsection{8. Roadmap \& Adoption
Strategy}\label{roadmap-adoption-strategy}}

\hypertarget{technical-and-product-development-timeline}{%
\subsubsection{8.1 Technical and Product Development
Timeline}\label{technical-and-product-development-timeline}}

Quantillon's roadmap is structured in three phases, balancing rapid
deployment with robust institutional integration. Each phase
incorporates clear milestones for TVL, user adoption, regulatory
preparation, and technological expansion.

\textbf{Phase 1 --- Three-Token Launch \& Bootstrap (Q4 2025)}

\begin{itemize}
\item
  Mainnet deployment of complete three-token ecosystem: QEURO, stQEURO,
  and QTI
\item
  aQEURO vault launch with Aave USDC integration as primary collateral
  backend
\item
  veQTI system activation with vote-escrow mechanics and governance
  layer implementation
\item
  Initial hedger onboarding and liquidity provisioning via Quantillon
  Labs
\item
  stQEURO auto-compounding mechanism launch with real-time yield
  calculation
\item
  First comprehensive audit reports published covering all three tokens
\item
  Bug bounty programs initiated with specialized focus on yield-bearing
  mechanics
\end{itemize}

Target Metrics:

\begin{itemize}
\item
  TVL: €10M total (€6M QEURO, €4M stQEURO)
\item
  Users: 10,000 total (7,000 QEURO holders, 3,000 stQEURO stakers)
\item
  Staking Ratio: 40\% of QEURO supply staked as stQEURO
\item
  QTI Market Cap: \$5M with 15\% governance participation
\item
  Swap Volume: €100M with 0.5\% average daily trading velocity
\end{itemize}

\textbf{Phase 2 --- Multi-Vault Expansion \& Institutional Integration
(Q1-Q2 2026)}

\begin{itemize}
\item
  Launch of mQEURO (MakerDAO) and bQEURO (Real World Assets) vault
  variants
\item
  Cross-chain deployment of complete three-token system on Base and
  Arbitrum
\item
  Strategic partnerships with CeDeFi brokers and institutional fintech
  platforms
\item
  Fiat on/off-ramp integrations with regulated custody partners for
  institutional access
\item
  Advanced stQEURO yield optimization strategies and institutional
  reporting features
\item
  Mobile application launch with neobank-style interface for retail
  adoption
\end{itemize}

Target Metrics:

\begin{itemize}
\item
  TVL: €100M total (€60M QEURO, €40M stQEURO across 3 vaults)
\item
  Users: 50,000 total with 60\% cross-chain adoption
\item
  Staking Ratio: 60\% of QEURO supply in stQEURO variants
\item
  QTI Market Cap: \$25M with 25\% governance participation
\item
  Institutional AUM: €20M in bQEURO institutional vault
\item
  Cross-Chain Volume: €1B with 30\% L2 distribution
\end{itemize}

\textbf{Phase 3 --- Advanced DeFi Integration \& Full Decentralization
(H2 2026-2027)}

\begin{itemize}
\item
  eQEURO vault launch with Ethena and advanced yield strategies (up to
  3x leverage)
\item
  Complete cross-chain ecosystem deployment (zkSync, Polygon, additional
  L2s)
\item
  DAO-controlled treasury activation with full Fee Switch implementation
\item
  Advanced institutional products: custom yield strategies, treasury
  management APIs
\item
  Full protocol decentralization with governance parameter control
  transferred to QTI holders
\item
  Integration with 50+ DeFi protocols supporting stQEURO as native
  collateral
\item
  Traditional finance bridges: CBDC compatibility layer and
  institutional custody integration
\end{itemize}

Target Metrics:

\begin{itemize}
\item
  TVL: €1B total across all vaults and chains
\item
  Users: 200,000+ with institutional treasury adoption
\item
  Staking Ratio: 70\% of QEURO supply optimized across vault variants
\item
  QTI Market Cap: \$100M+ with 45\% governance participation
\item
  Protocol Revenue: €10M+ annually with 80\% margin sustainability
\item
  Cross-Protocol Integration: stQEURO accepted in 50+ major DeFi
  platforms
\end{itemize}

\hypertarget{strategic-kpis-and-milestones}{%
\subsubsection{8.2 Strategic KPIs and
Milestones}\label{strategic-kpis-and-milestones}}

Quantillon tracks performance via a set of quantitative and qualitative
indicators aligned with its mission:

\begin{itemize}
\item
  \textbf{TVL Growth}: From €10M (post-launch) to €1B+ in 24 months.
\item
  \textbf{Swap Volume}: 20× TVL target maintained via market
  integrations and partner flows.
\item
  \textbf{User Base}: Growth from early DeFi adopters to mainstream EU
  retail and B2B.
\item
  \textbf{Protocol Revenue}: \textgreater€500K/year sustainable at €20M
  TVL.
\item
  \textbf{Vault Diversity}: Four vaults live with differentiated
  collateral and risk models.
\end{itemize}

Progress is governed by data transparency, with quarterly performance
reports published on-chain and by the Quantillon Foundation.

\hypertarget{ecosystem-development-and-network-effects}{%
\subsubsection{8.3 Ecosystem Development and Network
Effects}\label{ecosystem-development-and-network-effects}}

Adoption is not only technical but also social and economic.
Quantillon's adoption strategy includes:

\begin{itemize}
\item
  \textbf{Community growth}: Incentivized programs for ambassadors,
  contributors, and early adopters.
\item
  \textbf{B2B pipelines}: Outreach to wealth managers, DAOs, family
  offices, and treasuries.
\item
  \textbf{Ecosystem grants}: For developers building QEURO-integrated
  products (e.g., wallets, cross-chain bridges, tax tools).
\item
  \textbf{Educational content}: Financial literacy campaigns focused on
  DeFi yields, euro-denominated finance, and MiCA.
\end{itemize}

Quantillon is not merely launching a product---it is fostering a
movement. The roadmap emphasizes responsible growth, composability, and
transparency to establish QEURO as the euro-native financial layer of
Web3.

\hypertarget{team-operational-structure}{%
\subsection{9. Team \& Operational
Structure}\label{team-operational-structure}}

Quantillon is driven by a multidisciplinary team combining decades of
experience in software engineering, decentralized finance,
macroeconomics, and digital asset management. The organizational design
reflects the layered responsibilities of protocol development,
governance stewardship, and institutional interfacing.

\hypertarget{founding-team}{%
\subsubsection{9.1 Founding Team}\label{founding-team}}

\textbf{Toni Cantarutti -- CEO} Toni brings over 15 years of experience
as a software architect specializing in core C++ and system-level
development. His career spans R\&D roles at Intuisphere, Orange Labs,
and Thermo Fisher Scientific. He is the founder of Benarius, a CeFi
euro-yield platform, which laid the groundwork for Quantillon. Toni
leads protocol design, team coordination, and strategic vision.

\textbf{Nicolas Bellengé -- CTO} With 20 years of experience in software
engineering and project leadership, Nicolas is an expert in Web2/Web3
stack integration. He is the CEO of NBTC SAS, a company focused on the
acquisition, resale, and management of crypto assets. At Quantillon,
Nicolas oversees the smart contract infrastructure, full-stack
development, and cybersecurity framework.

\hypertarget{core-development-product-team}{%
\subsubsection{9.2 Core Development \& Product
Team}\label{core-development-product-team}}

Quantillon\textquotesingle s core team includes two full-stack
blockchain developers specializing in Solidity, DeFi integrations, and
protocol-level testing. The product division includes a UX/UI designer
and a communications strategist responsible for user onboarding,
documentation, and market engagement.

This lean structure ensures rapid iteration while maintaining rigorous
technical standards. Team roles are documented on-chain for transparency
and incentivized through long-term vesting in \$QTI tokens.

\hypertarget{organizational-entities}{%
\subsubsection{9.3 Organizational
Entities}\label{organizational-entities}}

To manage legal exposure, regulatory dialogue, and decentralized
operations, Quantillon is structured into three synergistic entities:

\begin{itemize}
\item
  \textbf{Quantillon Protocol} (on-chain): Fully decentralized smart
  contracts governed by \$QTI holders. Responsible for minting, vault
  logic, hedging infrastructure, and governance proposals.
\item
  \textbf{Quantillon Labs} (development): A legal entity responsible for
  developing and launching the protocol. Initially acts as a liquidity
  provider and hedger, remunerated by the protocol. Labs ensures code
  security, DevOps, and integrations.
\item
  \textbf{Quantillon Foundation} (compliance): A non-profit Swiss-style
  foundation that interfaces with regulators, auditors, and legal
  stakeholders. It safeguards multisig access, publishes disclosures,
  and facilitates DAO transitions.
\end{itemize}

This tripartite model preserves decentralization while ensuring
institutional credibility and legal defensibility---especially relevant
under MiCA's evolving guidance.

Quantillon's human capital strategy focuses on high-leverage
contributors, external auditors, and community-aligned governance. The
team is well-positioned to evolve into a DAO-governed protocol with
scalable institutional interfaces.

\hypertarget{appendices-techniques}{%
\subsection{10. Appendices Techniques}\label{appendices-techniques}}

\hypertarget{smart-contract-architecture}{%
\subsubsection{10.1 Smart Contract
Architecture}\label{smart-contract-architecture}}

Quantillon's codebase is structured around a modular Solidity framework,
designed for upgradability, auditability, and composability. Core
contracts include:

\begin{itemize}
\item
  \textbf{MintManager:} Handles the conversion of ERC20 assets to QEURO
  at oracle rates, manages collateral routing.
\item
  \textbf{VaultEngine:} Oversees collateral deposits, DeFi deployment,
  liquidation triggers, and hedger margins.
\item
  \textbf{YieldController:} Applies yield share logic, calculates Yield
  Shift, and manages protocol fees.
\item
  \textbf{GovernanceModule:} Administers voting logic, fee toggles, and
  whitelisting of new vaults or oracle feeds.
\end{itemize}

All contracts are non-custodial, with critical functions gated through
governance time-locks. Emergency pause mechanisms are included.

\hypertarget{oracle-and-pricing-infrastructure}{%
\subsubsection{10.2 Oracle and Pricing
Infrastructure}\label{oracle-and-pricing-infrastructure}}

The protocol relies on Chainlink as its primary data oracle, with
fallback nodes under Quantillon Labs. Oracle feeds are used for:

\begin{itemize}
\item
  EUR/USD spot rates
\item
  USDC price stability checks
\item
  On-chain collateral valuation (Aave, Maker)
\end{itemize}

A three-tier logic is applied:

\begin{enumerate}
\def\labelenumi{\arabic{enumi}.}
\item
  \textbf{Medianized price feeds} across multiple sources
\item
  \textbf{Heartbeat and deviation checks} to detect anomalies
\item
  \textbf{Fallback escalation} using governance-mandated oracles
\end{enumerate}

This ensures robust resistance to manipulation or downtime.

\hypertarget{yield-shift-formula}{%
\subsubsection{10.3 Yield Shift Formula}\label{yield-shift-formula}}

The dynamic yield redistribution follows a parametric curve governed by
the ratio of hedger supply to QEURO demand.

Let:

\begin{itemize}
\item
  \textbf{R} = net available yield from collateral
\item
  \textbf{H} = \% of target hedger pool filled
\item
  \textbf{Yh} = yield allocated to hedgers
\item
  \textbf{Yu} = yield allocated to users
\end{itemize}

Then:

\begin{itemize}
\item
  \textbf{Yh = min(1\%, R × H × α)}
\item
  \textbf{Yu = R − Yh − protocol fee (10\%)}
\end{itemize}

Where α is a governance-defined coefficient that tunes market
incentives. This creates an implicit rebalancing of protocol incentives
and peg resilience.

\hypertarget{value-at-risk-and-stress-testing}{%
\subsubsection{10.4 Value at Risk and Stress
Testing}\label{value-at-risk-and-stress-testing}}

Quantillon incorporates continuous monitoring of its financial health
using:

\begin{itemize}
\item
  \textbf{VaR (95\%/99\%)} models on EUR/USD and TVL drawdowns
\item
  \textbf{Stress tests} assuming FX shock (+/- 10\%), collateral yield
  drop to 0\%, or Aave liquidity collapse
\item
  \textbf{Daily simulations} to validate liquidation thresholds and
  hedger margin adequacy
\end{itemize}

Results are logged on-chain and reviewed by both internal risk modules
and external auditors. Public dashboards will visualize these metrics
post-launch.

\hypertarget{compliance-interfaces}{%
\subsubsection{10.5 Compliance Interfaces}\label{compliance-interfaces}}

Though out-of-scope under MiCA, the protocol maintains a
compliance-oriented posture:

\begin{itemize}
\item
  \textbf{Public transparency} of all vault mechanics, treasury flows,
  and governance changes
\item
  \textbf{Voluntary disclosures} issued by Quantillon Foundation
\item
  \textbf{Audit trails} for smart contracts and FX margin mechanisms
\end{itemize}

This hybrid approach supports future institutional integrations and
legal adaptability across EU jurisdictions.

Quantillon's technical backbone is crafted not only for security and
functionality, but also for auditability, regulatory interface, and
long-term evolvability.

\textbf{sparency} of all vault mechanics, treasury flows, and governance
changes

\begin{itemize}
\item
  \textbf{Voluntary disclosures} issued by Quantillon Foundation
\item
  \textbf{Audit trails} f
\end{itemize}
